% Created 2020-10-21 水 10:11
\documentclass{article}
\usepackage[utf8]{inputenc}
\usepackage[T1]{fontenc}
\usepackage{fixltx2e}
\usepackage{graphicx}
\usepackage{longtable}
\usepackage{float}
\usepackage{wrapfig}
\usepackage{rotating}
\usepackage[normalem]{ulem}
\usepackage{amsmath}
\usepackage{textcomp}
\usepackage{marvosym}
\usepackage{wasysym}
\usepackage{amssymb}
\usepackage{hyperref}
\tolerance=1000
\author{hisanobu-nakamura}
\date{\textit{<2019-06-27 木>}}
\title{Notes on "Statistics for Mathematicians" by Victor M. Panaretos}
\hypersetup{
  pdfkeywords={},
  pdfsubject={},
  pdfcreator={Emacs 25.3.2 (Org mode 8.2.10)}}
\begin{document}

\maketitle
Title: Binomial Distribution
Date: 2020-10-21
Category: math
Tags: math, statistics


\section{Chapter 1: Regular Probability Models}
\label{sec-1}

\begin{quote}
\textbf{Definition 1.1: Regular Parametric Probability Models}\\
\begin{itemize}
\item $X$ : $\mathbb{R}$-valued random variable
\item $F_{\theta}$ : distribution function of $X$
\item $\theta$ : a parameter in $\Theta \subseteq \mathbb{R}^{p}$ (parameter space)
\end{itemize}

The probability model $\{F_{\theta} : \theta \in \Theta\}$ will be calld \emph{\uline{regular}} if one of the two following conditions holds:

\begin{enumerate}
\item $\forall \theta  \in \Theta$, the distribution $F_{\theta}$ is continuous with density $f(x; \theta)$
\item $\forall \theta  \in \Theta$, the distribution $F_{\theta}$ is discrete with probability mass function $f(x;\theta)$ such that $\sum_{x \in \mathbb{Z}} f(x;\theta) = 1$ for all $\theta \in \Theta$.
\end{enumerate}
\end{quote}

\begin{itemize}
\item The model $F_{\theta}$ cannot switch between continuous and discrete depending on the value of $\theta$.
\item $\mathcal{X} := \{x \in \mathbb{R}: f(x;\theta) > 0 \}$ is called the \emph{\uline{sample space}} of $X$.
\end{itemize}

\subsection{Discrete Regular Models}
\label{sec-1-1}
\begin{quote}
\textbf{Definition 1.5 Binomial distribution}\\
A random variable $X$ is said to follow the binomial distribution with parameters $p \in (0,1)$ and $n \in \mathbb{N}$, denoted $X \sim Binom(n,p)$, if

\begin{enumerate}
\item $\mathcal{X} = \{0, \ldots ,n\}$,
\item $\displaystyle f(x;n,p) = \left( n \atop p\right) \binom{n}{p}p^{x}(1-p)^{n-x}$. $\displaystyle f(x;n,p) = \left( n \atop p\right) p^{x}(1-p)^{n-x}$.
\end{enumerate}
\end{quote}

\begin{itemize}
\item The mean:  $\mathbb{E}[X] = np$
\item The variance:  $Var[X] = np(1-p)$
\item The moment generating function:  $M(t) = (1- p + pe^{t})^{n}$
\end{itemize}

\textbf{Derivations}\\
The mean:
\begin{eqnarray}
\mathbb{E}[X]  & = & \sum_{x=0}^{n} x \binom{n}{x}p^{x}(1-p)^{n-x}\\
               & = & \sum_{x=1}^{n} x \binom{n}{x} p^{x}(1-p)^{n-x}\\
               & = & p \sum_{x=1}^{n} x \binom{n}{x} p^{x-1}(1-p)^{n-x}
\end{eqnarray}
Consider the following polynomial's derivative with respect to $P$
\begin{eqnarray}
(P + Q)^{n} & = & \sum_{x=0}^{n} \binom{n}{x}P^{x}Q^{n-x}\\
\frac{\partial}{\partial P}(P + Q)^{n} & = & n(P + Q)^{n-1} = \sum_{x=0}^{n} x \binom{n}{x}P^{x-1}Q^{n-x} \\
                                       & = & \frac{1}{P} \left( \sum_{x=0}^{n} x \binom{n}{x}P^{x}Q^{n-x}\right)\\
 nP(P + Q)^{n-1} & = &  \sum_{x=0}^{n} x \binom{n}{x}P^{x}Q^{n-x}
\end{eqnarray}
Putting $P=p$ and $Q=1-p$, we have
\begin{equation}
\mathbb{E}[X] = pn(p+1-p)^{n-1} = pn
\end{equation}
Th vatiance:
\begin{eqnarray}
Var[X] & = & \mathbb{E}[X^{2}] - \mathbb{E}[X]^{2}\\
       & = & \sum_{x=0}^{n} x^{2} \binom{n}{x}p^{x}(1-p)^{n-x} - n^{2}p^{2}
\end{eqnarray}
To calculate the first term in the last line, let us consider the following second derivative of the polynomial $(P+Q)^{n}$
\begin{eqnarray}
\frac{\partial^{2}}{\partial P^{2}}(P + Q)^{n} & = & n(n-1)(P + Q)^{n-2} \nonumber\\ 
                                               & = & \sum_{x=2}^{n} x(x-1) \binom{n}{x}P^{x-2}Q^{n-x} \nonumber\\
                                               & = & \sum_{x=2}^{n} x^{2} \binom{n}{x}P^{x-2}Q^{n-x} - \sum_{x=2}^{n} x \binom{n}{x}P^{x-2}Q^{n-x} \nonumber\\
                                               & = & \frac{1}{P^{2}} \left( \sum_{x=2}^{n} x^{2} \binom{n}{x}P^{x}Q^{n-x} - \sum_{x=2}^{n} x \binom{n}{x}P^{x}Q^{n-x} \right) \nonumber\\
                                               & = & \frac{1}{P^{2}} \left( \sum_{x=1}^{n} x^{2} \binom{n}{x}P^{x}Q^{n-x} - \sum_{x=1}^{n} x \binom{n}{x}P^{x}Q^{n-x} \right) \nonumber\\
                                               & = & \frac{1}{P^{2}} \left( \sum_{x=1}^{n} x^{2} \binom{n}{x}P^{x}Q^{n-x} - nP(P+Q)^{n-1} \right) \nonumber\\
\therefore \sum_{x=1}^{n} x^{2} \binom{n}{x}P^{x}Q^{n-x} & = & n(n-1)P^{2}(P + Q)^{n-2} + nP(P+Q)^{n-1} \nonumber
\end{eqnarray}
Putting $P=p$ and $Q=1-p$, we have
\begin{equation}
\sum_{x=0}^{n} x^{2} \binom{n}{x}p^{x}(1-p)^{n-x} & = & n(n-1)p^{2} + np\\
\end{equation}
Hence
\begin{equation}
Var[X] = n(n-1)p^{2} + np - n^{2}p^{2} = np(1-p)
\end{equation}
The moment generating function:
\begin{eqnarray}
M(t) = \mathbb{E}[e^{tX}] & = & \sum_{x=0}^{n} e^{tx} \binom{n}{x}p^{x}(1-p)^{n-x} \\
& = & \sum_{x=0}^{n}  \binom{n}{x} (pe^{t})^{x}(1-p)^{n-x} \\
& = & (1-p +pe^{t})^{n}
\end{eqnarray}
% Emacs 25.3.2 (Org mode 8.2.10)
\end{document}