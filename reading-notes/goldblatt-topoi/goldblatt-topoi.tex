% Created 2021-04-17 土 12:00
% Intended LaTeX compiler: pdflatex
\documentclass{article}
\usepackage[utf8]{inputenc}
\usepackage[T1]{fontenc}
\usepackage{graphicx}
\usepackage{grffile}
\usepackage{longtable}
\usepackage{wrapfig}
\usepackage{rotating}
\usepackage[normalem]{ulem}
\usepackage{amsmath}
\usepackage{textcomp}
\usepackage{amssymb}
\usepackage{capt-of}
\usepackage{hyperref}
\usepackage[margin=1.0in]{geometry}
\usepackage{tikz-cd}
\author{hisanobu-nakamura}
\date{\textit{<2019-06-27 木>}}
\title{Notes on "Topoi" by Goldblatt}
\hypersetup{
 pdfauthor={hisanobu-nakamura},
 pdftitle={Notes on "Topoi" by Goldblatt},
 pdfkeywords={},
 pdfsubject={},
 pdfcreator={}, 
 pdflang={English}}
\begin{document}

\maketitle
\tableofcontents


\section{Chapter 3: Arrows instead of epsilons}
\label{sec:org7fdf1f2}
\subsection{3.1. Monic arrows}
\label{sec:orge1a2464}
An arrow \(f:a \rightarrow b\) in a category \(\mathcal{C}\) is  \emph{monic} in \(\mathcal{C}\) if for any pair \(g,h:c \rightrightarrows a\) of \(\mathcal{C}\) -arrows, equality \(f\circ g = f \circ h\) implies that \(g=h\).
The symbolism \(f:a \rightarrowtail b\) is used to indicate that \(f\) is monic.
\subsubsection{3.1. Exercise Solutions}
\label{sec:orgfe2eec1}
In any category \\
(1) \(g \circ f\) is monic if both \(f\) and \(g\) are monic.\\
(2) If \(g \circ f\) is monic, then so is \(f\).\\
\textbf{Solutions} \\
(1) \\
If \((g \circ f) \circ h_{1} = (g \circ f) \circ h_{2}\), then 
\begin{eqnarray*}
(g \circ f) \circ h_{1} &=& (g \circ f) \circ h_{2}\\
g \circ (f \circ h_{1}) &=& g \circ (f \circ h_{2})\\
f \circ h_{1} &=& f \circ h_{2}  \quad (\because \text{g is monic})\\
h_{1} &=& h_{2}  \quad (\because \text{f is monic})
\end{eqnarray*}
Therefore, \((g \circ f)\) is monic. \\
(2) \\
If \(f \circ h_{1} = f \circ h_{2}\), then 
\begin{eqnarray*}
g \circ (f \circ h_{1}) &=& g \circ (f \circ h_{2})\\
(g \circ f) \circ h_{1} &=& (g \circ f) \circ h_{2}\\
h_{1} &=& h_{2}  \quad (\because \text{$f \circ g$ is monic})
\end{eqnarray*}
Hence, \(f\) is monic.

\subsection{3.2. Epic arrows}
\label{sec:org663d4e1}
An arrow \(f: a \rightarrow b\) is \emph{epic} (right-cancellable) in a category \(\mathcal{C}\) if for any pair \(g,h:b \rightrightarrows c\), the equality \(g \circ f = h \circ f\) implies \(g=h\) i.e. whenever a diagram
\[ \begin{tikzcd}
a \arrow{r}{f} \arrow[swap]{d}{f} & b \arrow{d}{g} \\%
b \arrow{r}{h}& c
\end{tikzcd}
\]
commutes, then \(g=h\). The notation \(f:a \twoheadrightarrow b\) is used for epic arrows.

\subsubsection{3.2. Exercises(Duals of exercises in 3.1)}
\label{sec:org359564e}
In any category\\
(1) \(g \circ f\) is epic if both \(f\) and \(g\) are epic.\\
(2) If \(g \circ f\) is epic, then so is \(g\).\\
\textbf{Solutions} \\
(1) \\
Suppose that \(f\) and \(g\) are both epic, then if \(h_{1} \circ (g \circ f) = h_{2} \circ (g \circ f)\),
\begin{eqnarray*}
h_{1} \circ (g \circ f) &=& h_{2} \circ (g \circ f)\\
(h_{1} \circ g) \circ f &=& (h_{2} \circ g) \circ f\\
h_{1} \circ f &=& h_{2} \circ f\\
h_{1} &=& h_{2}.
\end{eqnarray*}
Hence, $g \circ f$ is epic.\\
(2)\\
Suppose \(g \circ f\) is epic. Then, if \(h_{1} \circ g = h_{2} \circ g\),
\begin{eqnarray*}
(h_{1} \circ g) \circ f &=& (h_{2} \circ g) \circ f\\
h_{1} \circ (g \circ f) &=& h_{2} \circ (g \circ f)\\
h_{1} &=& h_{2}.
\end{eqnarray*}
Hence, \(g\) is epic.
\subsection{3.8 Products}
\label{sec:org0d5709d}
\textbf{DEFINITION}
A product in a category \(\mathcal{C}\) of two objects \(a\) and \(b\) is a \(\mathcal{C}\) -object \(a \times b\)
together with a pair (\(pr_{a}:a\times b \rightarrow a\), \(pr_{b}:a\times b \rightarrow b\)) of \(\mathcal{C}\) -arrows such that for any pair of \(\mathcal{C}\) -arrows of the form (\(f:c \rightarrow a\), \(g:c \rightarrow b\))
there is exactly one arrow \(\langle a,g \rangle : c \rightarrow a \times b\) making
\[ \begin{tikzcd}
                    && \arrow{lldd}[swap]{f} c \arrow[dashed]{dd}{\langle f,g\rangle} \arrow[rrdd,"g"]  && \\
		    &&&&\\
a  && \arrow{ll}[swap, near head]{pr_{a}} a \times b \arrow[rr, "pr_{b}"]&& b
\end{tikzcd}
\]
commute, i.e. such that  \(pr_{a}\circ \langle f,g \rangle = f\), \(pr_{b}\circ \langle f,g \rangle = g\). \(\langle f,g \rangle\) is the product arrow of \(f\) and \(g\) with respect to the projections \(pr_{a}\), \(pr_{b}\).

\end{document}