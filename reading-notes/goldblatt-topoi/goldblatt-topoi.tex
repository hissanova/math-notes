% Created 2020-09-29 火 11:55
\documentclass{article}
\usepackage[utf8]{inputenc}
\usepackage[T1]{fontenc}
\usepackage{fixltx2e}
\usepackage{graphicx}
\usepackage{longtable}
\usepackage{float}
\usepackage{wrapfig}
\usepackage{rotating}
\usepackage[normalem]{ulem}
\usepackage{amsmath}
\usepackage{textcomp}
\usepackage{marvosym}
\usepackage{wasysym}
\usepackage{amssymb}
\usepackage{hyperref}
\tolerance=1000
\usepackage[margin=1.0in]{geometry}
\usepackage{tikz-cd}
\author{hisanobu-nakamura}
\date{\textit{<2019-06-27 木>}}
\title{Notes on "Topoi" by Goldblatt}
\hypersetup{
  pdfkeywords={},
  pdfsubject={},
  pdfcreator={Emacs 25.3.2 (Org mode 8.2.10)}}
\begin{document}

\maketitle
\tableofcontents


\section{Chapter 3: Arrows instead of epsilons}
\label{sec-1}

\subsection{Exercise Solutions}
\label{sec-1-1}

\subsubsection{3.1.}
\label{sec-1-1-1}
In any category
\begin{enumerate}
\item $g \circ f$ is monic if both $f$ and $g$ are monic.
\item If $g \circ f$ is monic, then so is $f$.
\end{enumerate}
Solutions \\
(1) \\
If $(g \circ f) \circ h_{1} = (g \circ f) \circ h_{2}$, then 

\begin{eqnarray*}
(g \circ f) \circ h_{1} &=& (g \circ f) \circ h_{2}\\
g \circ (f \circ h_{1}) &=& g \circ (f \circ h_{2})\\
f \circ h_{1} &=& f \circ h_{2}  \quad (\because \text{g is monic})\\
h_{1} &=& h_{2}  \quad (\because \text{f is monic})
\end{eqnarray*}
Therefore, $(g \circ f)$ is monic. \\
(2) \\
If $f \circ h_{1} = f \circ h_{2}$, then 
\begin{eqnarray*}
g \circ (f \circ h_{1}) &=& g \circ (f \circ h_{2})\\
(g \circ f) \circ h_{1} &=& (g \circ f) \circ h_{2}\\
h_{1} &=& h_{2}  \quad (\because \text{$f \circ g$ is monic})
\end{eqnarray*}
Hence, $f$ is monic.
% Emacs 25.3.2 (Org mode 8.2.10)
\end{document}