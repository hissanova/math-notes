% Created 2019-06-22 土 17:05
\documentclass{article}
\usepackage[utf8]{inputenc}
\usepackage[T1]{fontenc}
\usepackage{fixltx2e}
\usepackage{graphicx}
\usepackage{longtable}
\usepackage{float}
\usepackage{wrapfig}
\usepackage{rotating}
\usepackage[normalem]{ulem}
\usepackage{amsmath}
\usepackage{textcomp}
\usepackage{marvosym}
\usepackage{wasysym}
\usepackage{amssymb}
\usepackage{hyperref}
\tolerance=1000
\author{hisanobu-nakamura}
\date{\textit{<2018-06-20 水>}}
\title{Square Root Difference Theorem}
\hypersetup{
  pdfkeywords={},
  pdfsubject={},
  pdfcreator={Emacs 25.3.2 (Org mode 8.2.10)}}
\begin{document}

\maketitle



\section{The Problem}
\label{sec-1}
Prove the following statetment:
$\forall n, k (>0) \in \mathbb{N}, \exists N \in \mathbb{N}$ such that
\begin{equation}
  \label{eq:main}
  (\sqrt{n+1} - \sqrt{n})^k = \sqrt{N+1} -\sqrt{N}
\end{equation}
\begin{equation}
  \label{eq:main}
  (\sqrt{n+1} - \sqrt{n})^k = \sqrt{N+1} -\sqrt{N}
\end{equation}


\textbf{Proof}

First, let us rephrase the statement in a more tactible way. Take logarithm of the L.H.S.
\begin{eqnarray*}
\ln (\sqrt{n + 1} - \sqrt{n})^k &=& k \ln (\frac{1}{\sqrt{n + 1} + \sqrt{n}}) \\
&=& - k \ln (\sqrt{n + 1} + \sqrt{n})
\end{eqnarray*}
Then, with $x = \sqrt{n}$, the logarithm at the last line is equal to $\sinh^{-1}{x}$. So, let us denote $\phi = - \sinh^{-1}{x}$.
Now, the statement reads that, for $\phi$ such that $\sinh^2 \phi$ is a natural number, prove that $\sinh^2 k \phi$ is also a natural number.
In fact, $\sinh k \phi$ can be expressed as a polynomial in terms of $\sinh{\phi} = \sqrt{n}$ and $\cosh{\phi} = \sqrt{n + 1}$ with integer coefficients. 
But in order for $\sinh^2 k \phi$ to be natural, $\sinh k \phi$ must factorize into a natural factor and an irrational factor such as $\sinh{\phi}$, $\cosh{\phi}$ or $\sinh{\phi}\cosh{\phi}$
The trigonometric-hyperbolic conversion rule:
\begin{equation}
  \label{eq:trig-hype}
  \cos{i \phi} = \cosh{\phi}, \quad \sin{i \phi} = i \sinh{\phi} 
\end{equation}
This leads to
\begin{eqnarray}
  \label{eq:binomial}
  (x + y)^{k} &=& \sum_{l + l^{\prime} = k} \binom{k}{l} x^l y^{l^{\prime}}
\end{eqnarray}
\begin{eqnarray}
  \label{eq:binomial-even}
  (x + y)^{2m} &=& x ^{2m} + \binom{2m}{1} \underline{x^{2m-1}y} + \binom{2m}{2} x^{2m-2}y^2 + \nonumber \\
               && \dots + \binom{2m}{2m - 2} x^2 y^{2m-2} + \binom{2m}{2m - 1} \underline{x y^{2m-1}} + y^{2m}
\end{eqnarray}
\begin{eqnarray}
  \label{eq:binomial-odd}
  (x + y)^{2m+1} &=& x ^{2m+1} + \binom{2m+1}{1} \underline{x^{2m}y} + \binom{2m+1}{2} x^{2m-1}y^2 + \nonumber \\
  && \dots + \binom{2m + 1}{2m - 1} \underline{x^2 y^{2m-1}} + \binom{2m + 1}{2m} x y^{2m} + \underline{y^{2m+1}}
\end{eqnarray}
% Emacs 25.3.2 (Org mode 8.2.10)
\end{document}