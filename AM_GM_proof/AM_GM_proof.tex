% Created 2020-02-24 月 22:49
\documentclass{article}
\usepackage[utf8]{inputenc}
\usepackage[T1]{fontenc}
\usepackage{fixltx2e}
\usepackage{graphicx}
\usepackage{longtable}
\usepackage{float}
\usepackage{wrapfig}
\usepackage{rotating}
\usepackage[normalem]{ulem}
\usepackage{amsmath}
\usepackage{textcomp}
\usepackage{marvosym}
\usepackage{wasysym}
\usepackage{amssymb}
\usepackage{hyperref}
\tolerance=1000
\usepackage[margin=1.0in]{geometry}
\usepackage{mymacros}
\usepackage{amsmath,amssymb,amsthm}
\author{hisanobu-nakamura}
\date{\textit{<2019-08-31 土>}}
\title{How to prove the arithmetic mean and geometric mean formula}
\hypersetup{
  pdfkeywords={},
  pdfsubject={},
  pdfcreator={Emacs 25.3.2 (Org mode 8.2.10)}}
\begin{document}

\maketitle
Title: A Proof for Arithmetic Mean and Geometric Mean Formula
Date: 2020-02-10
Category: math
Tags: math

\section{Arithmetic Mean and Geometric Mean Formula}
\label{sec-1}
\subsection{The Statement}
\label{sec-1-1}
Here is the arithmetic and geometric formula
\begin{equation*}
\label{ }
\frac{a_1 + \cdots + a_n}{n} \ge \sqrt[n]{a_1\cdots a_n},
\end{equation*}
where $\forall i \; a_i > 0$.

\subsection{Proof of the Statement}
\label{sec-1-2}
OK for $n=1$. Equality holds. Suppose that the inequality holds for all natural numbers up to $n$, then
\begin{eqnarray*}
a_1 + \cdots + a_{n+1} & = & \frac{a_2 + a_3 \cdots + a_{n+1}}{n} + \frac{a_1 + a_3 + \cdots + a_{n+1}}{n} + \cdots  \nonumber \\
                       &  &  + \frac{a_1 + a_2 + \overset{i}{\breve{\cdots} } + a_{n+1}}{n} + \cdots + \frac{a_1 + \cdots + a_{n}}{n}\nonumber\\
                       & = &  \sqrt[n]{a_2\cdots a_{n+1}} + \cdots + \sqrt[n]{a_1\cdots a_n} \nonumber \\
                       & = &  \sqrt[n+1]{a_1\cdots a_{n+1}} \left( \frac{\sqrt[n]{a_2\cdots a_{n+1}}}{\sqrt[n+1]{a_1\cdots a_{n+1}}} + \cdots + \frac{\sqrt[n]{a_1\cdots a_n}}{\sqrt[n+1]{a_1\cdots a_{n+1}}} \right) \nonumber \\
                       & = &  \sqrt[n+1]{a_1\cdots a_{n+1}} \left( \sqrt[n+1]{\frac{\sqrt[n]{a_2\cdots a_{n+1}}}{a_1}} + \cdots + \sqrt[n+1]{\frac{\sqrt[n]{a_1\cdots a_{n}}}{a_{n+1}}}\right) \nonumber \\
                       & = &  \sqrt[n+1]{a_1\cdots a_{n+1}} \left( \sqrt[n+1]{\frac{\alpha_1}{a_1}} + \cdots + \sqrt[n+1]{\frac{\alpha_{n+1}}{a_{n+1}}}\right) 
\end{eqnarray*}
where $\alpha_i=\sqrt[n]{a_1\overset{i}{\breve{\cdots}} a_{n+1}}$ (Here, $\overset{i}{\breve{\cdots}}$ means omitting the i-th term in the sum or the product). It is easy to see that
\begin{equation*}
\label{ }
\sqrt[n+1]{\frac{\alpha_1}{a_1}}  \cdots  \sqrt[n+1]{\frac{\alpha_{n+1}}{a_{n+1}}} = 1.
\end{equation*}
So, if we can show the following proposition, we are done.\\
\textbf{\emph{Lemma}}:
 If $b_1, \cdots, b_{n+1} >0$ and $b_1 \cdots b_{n+1} = 1$, then
\begin{equation}
\label{AM_GM_normalised}
b_1 + \cdots + b_{n+1} \ge n+1
\end{equation}
\begin{right}
$\qed$
\end{right}

Now, by substituting $b_i = \sqrt[n+1]{\frac{\alpha_i}{a_i}}$, it is straightforward to see that
\begin{eqnarray*}
a_1 + \cdots + a_{n+1} & = &  \sqrt[n+1]{a_1\cdots a_{n+1}} \left( \sqrt[n+1]{\frac{\alpha_1}{a_1}} + \cdots + \sqrt[n+1]{\frac{\alpha_{n+1}}{a_{n+1}}}\right) \nonumber \\
                       & \ge &  \sqrt[n+1]{a_1\cdots a_{n+1}} (n + 1 ) \nonumber \\
\end{eqnarray*}

\begin{right}
$\qed$
\end{right}
\subsection{Proof of the Lemma}
\label{sec-1-3}
\textbf{\emph{Lemma}}:
 If $b_1, \cdots, b_{n+1} >0$ and $b_1 \cdots b_{n+1} = 1$, then
\begin{equation}
\label{AM_GM_normalised}
b_1 + \cdots + b_{n+1} \ge n+1
\end{equation}
\textbf{\emph{Proof}}:
 First, if $b_1 = \cdots = b_{n+1} =1$, then the equality holds.
We notice that $\exists i$ such that $b_{i} > 1$ implies that $\exists j$ such that $b_{j} < 1$.
So, let us assume the condition and proceed to prove the statement by induction. 
For $n=1$, we can assume that $b_1 = 1-c_1$ with $0 < c_1 <1$. Then
\begin{equation*}
\label{ }
b_1 + b_2 = b_1 + \frac{1}{b_1} = 1 - c_1 + \frac{1}{1-c_1} \ge 1-c_1 + 1 + c_1 = 2.
\end{equation*}
Now, let us suppose that (\ref{AM_GM_normalised}) holds true for all the natural numbers up to n. Regarding the fact mentioned above, let us suppose that $b_n = 1 + c_n$ and $b_{n+1} = 1+c_{n+1}$ with $c_n < 0$ and $c_{n+1} > 0$. Then, by the assumption, 
\begin{equation*}
\label{ }
(b_1 \cdots b_{n-1})(b_n b_{n+1}) = 1 \quad \text{implies} \quad b_1 + \cdots + b_{n-1} + b_n b_{n+1} \ge n.
\end{equation*}
If we can say
\begin{equation*}
\label{ }
b_1 + \cdots + b_{n-1} + (b_n + b_{n+1}) \ge \text{or} > b_1 + \cdots b_{n-1} + (b_n b_{n+1} + 1),
\end{equation*}
then the inequality follows. But
\begin{equation*}
\label{ }
b_n + b_{n+1} - (b_n b_{n+1} + 1) = -c_n c_{n+1} > 0.
\end{equation*}
Hence, it follows that 
\begin{equation*}
\label{ }
b_1 + \cdots + b_{n-1} + b_n + b_{n+1} > b_1 + \cdots + b_{n-1} + b_n b_{n+1} + 1 \ge n + 1
\end{equation*}
\begin{right}
$\qed$\\
\end{right}
% Emacs 25.3.2 (Org mode 8.2.10)
\end{document}