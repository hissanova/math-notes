\documentclass[a4j,12pt]{jarticle}

\usepackage[margin=0.5in]{geometry}
\usepackage{ascmac}
\usepackage{amsmath}
\usepackage{amssymb}
\usepackage{amsthm}
\usepackage{fancybox}
\usepackage[dvips,final]{graphicx}
\usepackage{tabularx}
\usepackage{enumerate}
\numberwithin{equation}{section}
\newtheorem{example}{例}[section]

\newcommand{\N}{\mathbb N}
\newcommand{\R}{\mathbb R}
\newcommand{\C}{\mathbb C}
\newcommand{\bra}[1]{\langle{#1}|}
\newcommand{\ket}[1]{|{#1}\rangle}
\newcommand{\Expect}[3]{\langle {#1}|{#2}|{#3} \rangle}
\newcommand{\dev}[2]{\langle (\Delta {#1})^2 \rangle _{{#2}}}
\newcommand{\sgm}[1]{\sigma_{#1}}
\newcommand{\sgmOp}[1]{\hat{\sigma}_{#1}}
\newcommand{\hfangle}[1]{{\frac{#1}{2}}}
\newcommand{\clmnV}[2]
		{\left(
			\begin{matrix}
			 {#1}\\
			 {#2}\\
		 	\end{matrix}
		\right)
		}
\newcommand{\clmnVsan}[3]
		{\left(
			\begin{matrix}
			 {#1}\\
			 {#2}\\
			 {#3}\\
		 	\end{matrix}
		\right)
		}
\newcommand{\mtrx}[4]
	{\left(
		\begin{matrix}
		 {#1} & {#2} \\
		 {#3} & {#4} \\
		\end{matrix}
	\right)
	}
\newcommand{\inn}[2]{\langle{#1}|{#2}\rangle}
\newcommand{\itbf}[1]{\textit{\textbf{#1}}}

\newtheorem{thm}{定理}[section]
\newtheorem{dfn}{定義}[section]
\newtheorem{prop}[thm]{命題}
\renewcommand{\proofname}{\textbf{証明}}

\begin{document}
\title{Pointless Topology 勉強ノート}
\date{2024年6月13日〜}
\author{中村仁宣}
\maketitle

\section{Preliminary}
\subsection{Topology トポロジー}
Let $\mathcal{P}(X)$ denote the power set of $X$.
\begin{dfn}[Topology トポロジー]
  A \itbf{topological space} is an ordered pair $(X, \tau), \; \tau \subseteq \mathcal{P}(X)$ which satisfies the following properties
  \begin{enumerate}
  \item $\varnothing \in \tau$ and $X \in \tau$.
  \item if $U,V \in \tau$, then $U \cap V \in \tau$.
  \item if $\forall I, U_i \in \tau \text{ forall } i \in I$, then $\bigcup_{i\in I}U_i$.
  \end{enumerate}
  $\tau$ is called the \itbf{topology} of $X$.
  The members of the topology $U\in\tau$ is said to be \itbf{open} and $V \subseteq X$ is said to be \itbf{closed} if $\exists U$ open such that $V = U^c$.
\end{dfn}
\begin{dfn}[Separation Axioms 分離公理]
  A space $(X,\tau)$ is called $T_i$ ,if respectively satisfies the following conditions,
  \begin{enumerate}
  \item $T_0$: $\forall x,y \in X$ $\exists$ an open set $U \in \tau$ such that $U$ contains one of $x,y$ and not the other.
  \item $T_1$: $\forall x,y \in X$ $\exists$ a nhood of each not containing the other.
  \end{enumerate}
\end{dfn}
\begin{example}[$T_0$-space]
  $X=\{a,b\}, \tau=\{\varnothing, \{a\}, X\}$
\end{example}
\subsection{Posets, Lattices 半順序集合、束}
\begin{dfn}[Posets]
  A \itbf{partial order (半順序)} on a set $X$ is a binary relation $R \subseteq X \times X$ satisfying,
  \begin{enumerate}
  \item $\forall a, aRa$ (reflexivity, 反射律),
  \item $\forall a,b,c,\; aRb \; \& \; bRc \Rightarrow aRc$ (transitivity, 推移律),
  \item $\forall a,b,\; aRb \; \& \; bRa \Rightarrow a=b$ (antisymmetry, 反対称律).
  \end{enumerate}
  if moreover
  \begin{enumerate}
    \setcounter{enumi}{3}
  \item $\forall a,b$ either $aRb$ or $bRa$ holds,
  \end{enumerate}
  it is said to be a \itbf{linear} or \itbf{total} order.\\
  A \itbf{poset} or \itbf{partially ordered set}, $(X, \le)$ is a set with a partial order.
  If the order of a poset is linear (or total), it is called a \itbf{linearly ordered set}, \itbf{totally ordered set} or \itbf{chain}.
  A relation that satisfies only (1) and (2) is called \itbf{preorder}.
\end{dfn}
\begin{dfn}[Suprema, infima]
  A \itbf{supremum} $s$ of a subset $M \subseteq (X, \le)$ the least upper bound of $M$, that is
  \begin{enumerate}
  \item $\forall m \in M, m \le s,$
  \item $\forall m \in M, m \le x \Rightarrow s \le x$.
  \end{enumerate}
  Similarly, a \itbf{infimum} of a subset $M \subseteq (X, \le)$ the greatest lower bound of $M$.\\
  We also call a supremum a \itbf{join} and an infimum \itbf{meet} and notate $\sup M, \inf M$ or $\bigvee M, \bigwedge M$ respectively.\\
  For finite cases, we wirte $a \vee b := \sup\{a,b\}$ or $a_1 \vee \dots \vee a_n := \sup\{a_1\dots a_n\}$ and $a \wedge b := \inf\{a,b\}$ or $a_1 \wedge \dots \wedge a_n := \inf\{a_1\dots a_n\}$.
\end{dfn}
Since each $x\in X$ is both a lower and an upper bound of the empty set $\varnothing$,
\begin{equation}
  \sup\varnothing \text{ is the least element of } X
\end{equation}
and
\begin{equation}
  \inf\varnothing \text{ is the greatest element of } X
\end{equation}
We use the symbols $0$ or $\bot$ for the former and $1$ or $\top$ for the latter.\\
\begin{dfn}[Semilattices, Lattice]
  A \itbf{meet-semilattice} is a poset $X$ such that $\forall a,b \in X$ there exists an infimum $a \wedge b$.\\
  A \itbf{join-semilattice} is a poset $X$ such that $\forall a,b \in X$ there exists an supremum $a \vee b$.\\
  A \itbf{lattice} is a poset $X$ such that $\forall a,b \in X$ both an infimum $a \wedge b$ and a supremum $a \vee b$ exist.\\
  A \itbf{bounded lattice} is a poset in which all finite subsets have infima and suprema (i.e. a lattice with bottom and top).\\
  A poset is a \itbf{complete lattice} if every subset has a supremum and an infimum.
\end{dfn}
In a bounded semilattice, $\wedge$ or $\vee$ is a binary operation and satisfies the following properties,
\begin{align}
  & a \wedge a = a && a \vee a = a\\
  & a \wedge b = b \wedge a &&  a \vee b = b \vee a \\
  & (a \wedge b) \wedge c = a \wedge (b \wedge c)  && (a \vee b) \vee c = a \vee (b \vee c) \\
  & a \wedge 1 = a && a \vee 0 = a.
\end{align}
In other words, bounded semilattices are commutative monoids (semigroup with unit/identity element) in which every element is idempotent.\\
\begin{thm}
  Let $(A, \vee, 0)$ be a commutative monoid in which every element is idempotent.
  Then there exists a unique partial order on $A$ such that $a \wedge b$ is the join of $a$ and $b$, and $0$ is the least element.
\end{thm}
\begin{proof}
  Clearly, if such a partial order exists,
  \begin{equation}
    a \le b \Leftrightarrow a \vee b = b.
  \end{equation}
  Indeed, it is 
\end{proof}
A lattice can also be defined purely algebraically in those terms,
\begin{dfn}
  A \itbf{lattice} $(L, \vee, \wedge)$ is an algebra (a set with two binary operations) that satisfy
  \begin{align*}
    \label{eq:lattice-def}
    & (L1) &  & a \wedge a = a && a \vee a = a                       &  & \text{(idempotency)}   \\
    & (L2) &  & a \wedge b = b \wedge a && a \vee b = b\vee a        &  & \text{(commutativity)} \\
    & (L3) &  & (a \wedge b) \wedge c = a \wedge (b \wedge c)       &  & (a \vee b) \vee c = a \vee (b\vee c)                 &  & \text{(associativity)} \\
    & (L4) &  & a \vee (a \wedge b) = a && a \wedge (a \vee  b) =  a &  & \text{(absorption identities)}
  \end{align*}
\end{dfn}
\begin{dfn}[Ideal]
  An \itbf{ideal} in a bounded distributive lattice $L$ is a subset $J \subseteq L$ such that
  \begin{eqnarray}
    \label{eq:ideal}
    && 0 \in J,\\
    && a,b \in J \Rightarrow a \vee b \in J,\\
    && b\le a \;\&\; a \in J \Rightarrow b \in J.
  \end{eqnarray}
\end{dfn}
\begin{dfn}[Filter]
  A \itbf{filter} in a bounded distributive lattice $L$ is a subset $F \subseteq L$ such that
  \begin{eqnarray}
    \label{eq:filter}
    && 1 \in F,\\
    && a,b \in F \Rightarrow a \wedge b \in F,\\
    && b \ge a \;\&\; a \in F \Rightarrow b \in F.
  \end{eqnarray}
\end{dfn}
\section{Stone Spaces}
\section{Spaces and Lattices of Open Sets}
We will suppose that all topological spaces that appear here will be $T_0$.
\subsection{Sober spaces}
\begin{dfn}[meet-irrducibility]
  Let $(X,\tau)$ be a top.space. $W\in \tau$ is said to be a \itbf{meet-irreducible} open set if $U,V\in \tau$ and $U \cap V \subseteq W$, then either $U \subseteq W$ or $V \subseteq W$.
\end{dfn}
\begin{dfn}[sober space]
  $X$ is said to be \itbf{sober} if all the meet-irreducible open sets are of the form $X\backslash \overline{\{x\}}$.
\end{dfn}
\begin{prop}
  Each Haudorff space is sober.
\end{prop}
\begin{proof}
  Suppose $W$ is meet-irreducible, for contradiction, there exists $x_1,x_2\notin W$ and $x_i \in U_i, x_j \notin U_i(i \ne j)$.
  Then $W = (W \cup U_1) \cap (W \cup U_2)$ and $W\cup U_i \nsubseteq W$.
\end{proof}

% ---REFERENCES---%
\begin{thebibliography}{10}
\bibitem{Picado}
  Jorge Picado, Ale\u{s} Putlr, Frames and Locales:Topology without points, Birkh\"auser.
\end{thebibliography}

\end{document}
