\documentclass[a4j,12pt]{jarticle}

\usepackage[margin=0.5in]{geometry}
\usepackage{ascmac}
\usepackage{amsmath}
\usepackage{amssymb}
\usepackage{amsthm}
\usepackage{fancybox}
\usepackage[dvips,final]{graphicx}
\usepackage{tabularx}
\usepackage{enumerate}
\numberwithin{equation}{section}
\newtheorem{example}{Example}[section]

\newcommand{\N}{\mathbb N}
\newcommand{\R}{\mathbb R}
\newcommand{\C}{\mathbb C}
\newcommand{\bra}[1]{\langle{#1}|}
\newcommand{\ket}[1]{|{#1}\rangle}
\newcommand{\Expect}[3]{\langle {#1}|{#2}|{#3} \rangle}
\newcommand{\dev}[2]{\langle (\Delta {#1})^2 \rangle _{{#2}}}
\newcommand{\sgm}[1]{\sigma_{#1}}
\newcommand{\sgmOp}[1]{\hat{\sigma}_{#1}}
\newcommand{\hfangle}[1]{{\frac{#1}{2}}}
\newcommand{\clmnV}[2]
		{\left(
			\begin{matrix}
			 {#1}\\
			 {#2}\\
		 	\end{matrix}
		\right)
		}
\newcommand{\clmnVsan}[3]
		{\left(
			\begin{matrix}
			 {#1}\\
			 {#2}\\
			 {#3}\\
		 	\end{matrix}
		\right)
		}
\newcommand{\mtrx}[4]
	{\left(
		\begin{matrix}
		 {#1} & {#2} \\
		 {#3} & {#4} \\
		\end{matrix}
	\right)
	}
\newcommand{\inn}[2]{\langle{#1}|{#2}\rangle}
\newcommand{\itbf}[1]{\textit{\textbf{#1}}}

\newtheorem{thm}{定理}[section]
\newtheorem{dfn}{定義}[section]
\newtheorem{prop}[thm]{proposition}

\begin{document}
\title{Pointless Topology 勉強ノート}
\date{2024年6月13日〜}
\author{中村仁宣}
\maketitle

\section{Preliminary}
\subsection{Topology トポロジー}
Let $\mathcal{P}(X)$ denote the power set of $X$.
\begin{dfn}[Topology トポロジー]
  A \itbf{topological space} is an ordered pair $(X, \tau), \; \tau \subseteq \mathcal{P}(X)$ which satisfies the following properties
  \begin{enumerate}
  \item $\varnothing \in \tau$ and $X \in \tau$.
  \item if $U,V \in \tau$, then $U \cap V \in \tau$.
  \item if $\forall I, U_i \in \tau \text{ forall } i \in I$, then $\bigcup_{i\in I}U_i$.
  \end{enumerate}
  $\tau$ is called the \itbf{topology} of $X$.
  The members of the topology $U\in\tau$ is said to be \itbf{open} and $V \subseteq X$ is said to be closed if $\exists U$ open such that $V = U^c$.
\end{dfn}
\begin{dfn}[Separation Axioms 分離公理]
  A space $(X,\tau)$ is called $T_i$ ,if respectively satisfies the following conditions,
  \begin{enumerate}
  \item $T_0$: $\forall x,y \in X$ $\exists$ an open set $U \in \tau$ such that $U$ contains one of $x,y$ and not the other.
  \item $T_1$: $\forall x,y \in X$ $\exists$ a nhood of each not containing the other.
  \end{enumerate}
\end{dfn}
\subsection{Posets, Lattices 半順序集合、束}
\begin{dfn}[Posets]
  A \itbf{partial order (半順序)} on a set $X$ is a binary relation $R \subseteq X \times X$ satisfying,
  \begin{enumerate}
  \item $\forall a, aRa$ (reflexivity, 反射律),
  \item $\forall a,b,c,\; aRb \; \& \; bRc \Rightarrow aRc$ (transitivity, 推移律),
  \item $\forall a,b,\; aRb \; \& \; bRa \Rightarrow a=b$ (antisymmetry, 反対称律).
  \end{enumerate}
  if moreover
  \begin{enumerate}
    \setcounter{enumi}{3}
  \item $\forall a,b$ either $aRb$ or $bRa$ holds,
  \end{enumerate}
  it is said to be a \itbf{linear} or \itbf{total} order.\\
  A \itbf{poset} or \itbf{partially ordered set}, $(X, \le)$ is a set with a partial order.
  If the order of a poset is linear (or total), it is called a \itbf{linearly ordered set}, \itbf{totally ordered set} or \itbf{chain}.
  A relation that satisfies only (1) and (2) is called \itbf{preorder}.
\end{dfn}

\section{Spaces and Lattices of Open Sets}
We will suppose that all topological spaces that appear here will be $T_0$.
% ---REFERENCES---%
\begin{thebibliography}{10}
\bibitem{Picado}
  Jorge Picado, Ale\u{s} Putlr, Frames and Locales:Topology without points, Birkh\"auser.
\end{thebibliography}

\end{document}