% Created 2019-08-31 土 17:19
\documentclass{article}
\usepackage[utf8]{inputenc}
\usepackage[T1]{fontenc}
\usepackage{fixltx2e}
\usepackage{graphicx}
\usepackage{longtable}
\usepackage{float}
\usepackage{wrapfig}
\usepackage{rotating}
\usepackage[normalem]{ulem}
\usepackage{amsmath}
\usepackage{textcomp}
\usepackage{marvosym}
\usepackage{wasysym}
\usepackage{amssymb}
\usepackage{hyperref}
\tolerance=1000
\author{hisanobu-nakamura}
\date{\textit{<2019-08-31 土>}}
\title{Elliptic and Theta Functions}
\hypersetup{
  pdfkeywords={},
  pdfsubject={},
  pdfcreator={Emacs 25.3.2 (Org mode 8.2.10)}}
\begin{document}

\maketitle
\tableofcontents


\section{About this note}
\label{sec-1}
This note contains notes for Kotan seminar on elliptic and theta functions.

\section{Examples of Elliptic Integrals}
\label{sec-2}
Here, we gather some examples varying from pure mathematics to physics
\subsection{Arc Length of Ellipse}
\label{sec-2-1}
Canonical form (標準形):
\begin{equation}
\frac{x^{2}}{a^{2}} +\frac{y^{2}}{b^{2}}=1 \quad with \; a \ge b.
\end{equation}

Parametric form
\begin{equation}
\mathbf{p} =\left(
\begin{array}{c}
 x \\
 y
\end{array} \right)
=\left(
\begin{array}{c}
 a \sin{\varphi} \\
 b \cos{\varphi}
\end{array} \right)
\;, where \; 0 \le \varphi < 2 \pi
\end{equation}
The length of a curve $\mathbf{p}(t) = (x(t), y(t))$ is
\begin{eqnarray}
s(u) &=& \int_{0}^{u} \left| \frac{d \mathbf{p}}{d t} \right| dt  \\ \nonumber
     &=& \int_{0}^{u} \sqrt{ \left( \frac{d x}{d t} \right)^{2} +  \left( \frac{d x}{d t} \right)^{2} } \; dt
\end{eqnarray}
For the ellipse's case, we have
\begin{eqnarray}
s(\varphi) &=& \int_{0}^{\varphi} \sqrt{ a^{2} \cos^{2}{\varphi} +  b^{2}  \sin^{2}{\varphi}} \; d\varphi \\ \nonumber
     &=& a \int_{0}^{u} \sqrt{ 1 -  k^{2}  \sin^{2}{\varphi}} \; d\varphi 
\end{eqnarray}
where we defined $k = \sqrt{\frac{a^2 - b^2}{a^2}}$.
\begin{eqnarray}
E(k, \varphi ) &:=& \int_{0}^{u} \sqrt{ 1 -  k^{2}  \sin^{2}{\varphi}} \; d\varphi \\
E(k) &:=& E(k,  \pi/2 )
\end{eqnarray}
$E(k, \varphi )$ is called the \textbf{second incomplete elliptic integral (第2種不完全楕円積分)} and $E(k)$ the \textbf{second complete elliptic integral (第2種完全楕円積分)}. The total length $s_{total}$ of the ellipse is given by 
\begin{equation}
s_{total} = 4a E(k) 
\end{equation}
Alternatively, putting \$y = \textpm{} b \sqrt\{1 - \frac\{x$^{\text{2}}$\}\{a$^{\text{2}}$\}\}, the length can also be obtained as
\begin{eqnarray}
s &=& \int_{0}^{x} \sqrt{ 1  +  \left( \frac{d x}{d t} \right)^{2} } \; dx \\ \nonumber
     &=& a \int_{0}^{u} \sqrt{ 1  +   \frac{b^{2}}{a^{2}} \frac{\frac{x^{2}}{a^{2}}}{1 - \frac{x^{2}}{a^{2}}} } \; dx
\end{eqnarray}
Let us substitute further with $z = \frac{x}{a}$, 
\begin{center}
\begin{tabular}{ll}
$x$ & $0 \rightarrow a$\\
\hline
$z$ & $0 \rightarrow 1$\\
\end{tabular}
\end{center}
% Emacs 25.3.2 (Org mode 8.2.10)
\end{document}